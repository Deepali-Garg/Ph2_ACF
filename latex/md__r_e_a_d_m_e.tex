Supposed to contain\-:


\begin{DoxyItemize}
\item A middleware A\-P\-I layer, implemented in C++, which will basically wrap to abstracted functions the firmware calls and handshakes currently hardcoded into D\-A\-Q systems software.
\item A C++ object-\/based library describing the system components (C\-B\-Cs, Hybrids, Boards) and their properties(values, status)
\end{DoxyItemize}

\subsection*{M\-C\-P Test Interface }

Here are the step to make the program functional


\begin{DoxyEnumerate}
\item Recover the Git\-Hub repo with the latest tested version of the M\-C\-P.
\item Change in \hyperlink{_definition_8h}{H\-W\-Description/\-Definition.\-h} the path to the u\-Hal connection file.
\begin{DoxyItemize}
\item It's per default set to \-: \href{file:///afs/cern.ch/user/n/npierre/public/settings/connections.xml}{\tt file\-:///afs/cern.\-ch/user/n/npierre/public/settings/connections.\-xml}
\item You have to change it to \-: \href{file://$(BUILD)/settings/connections.xml}{\tt file\-://\$(\-B\-U\-I\-L\-D)/settings/connections.\-xml} (where  is the path to the root of the Git\-Hub repo you recovered)
\end{DoxyItemize}
\item Do a make in the root the repo (make sure you have all u\-Hal, root, boost... libraries on your computer)
\item Launch testpgrm command if you want to test if everything is working good
\item Launch mcp to play with the Test Interface
\end{DoxyEnumerate}

\subsection*{What can you do with the software ? }

A Glib is created per default (maybe in the future you will be able to play with more than one Glib)

You can then add Modules to the Glib and Cbcs to the Modules you created. When creating a Cbc, you can choose the config file you will load to its register map. If you want to add your personal config file, please make sure to add it in \#define in the H\-W\-Description/\-Description.\-h and then to add an option in the M\-C\-P code.

After this creation round, you can do anything you want \-:
\begin{DoxyItemize}
\item Configure the Glib or the Cbcs
\item Manipulate the registers in the Glib
\item Manipulate the registers in the Cbcs
\item Do the manipulation to Start/\-Stop an acquisition, but not actually acquiring something relevant
\item Have a configuration recap of all the objects you created
\end{DoxyItemize}

Concerning the manipulation of the Cbcs, you have the opportunity to modify all the Cbcs of a same Module at once with the Broadcast feature of the Cbc.

When you write a register in the Glib or the Cbc, the writing is updated to the map contained in Glib/\-Cbc objects so that you're always fully aware of what is in the Glib/\-Cbc register.

For writing value in register, we invite you to put in the following format \-: 0x\-\_\-\-\_\-. You must thus type '0x\-F\-F' for exemple and not just 'F\-F' in the command line. That might change in the future..

At the end of the program, the register maps are saved into output\-\_\-\-X\-\_\-\-X.\-txt files you can find in the settings directory. For example, output\-\_\-1\-\_\-2.\-txt contains the register map of the Cbc 2 of the Module 1.

More features will be implemented later, so make sure to have the last release locally to benefit from them.

Warning ! \-: be careful with options choice in the program menus, some mistypes can leed to unexpected hazards \-:-\/(.

\subsection*{Last Updates }


\begin{DoxyItemize}
\item 09/07/14 \-: Added threading for stack writing registers
\item 10/07/14 \-: Read Data from acquisition in a rubbish format
\end{DoxyItemize}

\subsection*{Future Improvements }


\begin{DoxyItemize}
\item Implementation of generic Be\-Board and Be\-Board\-Interface Class which will help us dealing with multiple board type (Glib, F\-C7).
\item Work on the Cbc class to make it recognize if the Cbc is from a Glib or a F\-C7.
\item Implement the new Strasbourg firmware as soon as it comes out.
\end{DoxyItemize}

\subsection*{Support, Suggestions ? }

For any support/suggestions, mail Lorenzo Bidegain, Nicolas Pierre or Georg Auzinger. 