~~~~~~{\bfseries Supposed to contain}


\begin{DoxyItemize}
\item A middleware A\-P\-I layer, implemented in C++, which will basically wrap to abstracted functions the firmware calls and handshakes currently hardcoded into D\-A\-Q systems software
\item A C++ object-\/based library describing the system components (C\-B\-Cs, Hybrids, Boards) and their properties(values, status)
\item The M\-C\-P test program which is the wrapping the previous two.
\end{DoxyItemize}

\subsection*{Different versions for different setups }

On this Git\-Hub, you can find different version of the software \-:
\begin{DoxyItemize}
\item An agnostic (to the number of C\-B\-Cs) version with the new structure in the Master branch
\item A 2\-C\-B\-C version on the 2\-C\-B\-C branch with the previous structure
\item A 8\-C\-B\-C version on the 8\-C\-B\-C branch with the previous structure
\item An in-\/progress version in the dev branch \par
 \par
 {\bfseries What are the differences between the 2\-C\-B\-C/8\-C\-B\-C versions ?}
\end{DoxyItemize}

The differences mainly resides in the size of the data buffer for the D\-A\-Q, when all the access to both Board and Cbc registers is done the same way. Also, some functions are present in 8\-C\-B\-C and not in 2\-C\-B\-C due to the fact that the firmware of the 8\-C\-B\-C is offering more possibilities of recovering infos from the Hardware (as the type of hardware for example)

\subsection*{The Test Software itself \-: the M\-C\-P Test Interface }

You'll find an install step by step and a How To. \par
 \par
 ~~~~~~{\bfseries Install the M\-C\-P Test Interface step by step...}

Here are the step to make the program functional


\begin{DoxyEnumerate}
\item Recover the Git\-Hub repo with the latest tested version of the M\-C\-P.
\item Change in \hyperlink{_definition_8h}{H\-W\-Description/\-Definition.\-h} the path to the u\-Hal connection file.
\begin{DoxyItemize}
\item It's per default set to \-: \href{file:///afs/cern.ch/user/n/npierre/dev/settings/connections.xml}{\tt file\-:///afs/cern.\-ch/user/n/npierre/dev/settings/connections.\-xml}
\item You have to change it to \-: \href{file://$(BUILD)/settings/connections.xml}{\tt file\-://\$(\-B\-U\-I\-L\-D)/settings/connections.\-xml} (where  is the path to the root of the Git\-Hub repo you recovered)
\end{DoxyItemize}
\item Change in settings/connections.\-xml the path to the adress table
\begin{DoxyItemize}
\item It's per default set to \-: \href{file:///afs/cern.ch/user/n/npierre/dev/settings/adress_table}{\tt file\-:///afs/cern.\-ch/user/n/npierre/dev/settings/adress\-\_\-table}(2\-C\-B\-C or 8\-C\-B\-C).xml
\item You have to change it to \-: \href{file://$(BUILD)/settings/adress_table}{\tt file\-://\$(\-B\-U\-I\-L\-D)/settings/adress\-\_\-table}(2\-C\-B\-C or 8\-C\-B\-C).xml (where  is the path to the root of the Git\-Hub repo you recovered)
\end{DoxyItemize}
\item Create /lib, /bin and /output directories in the root of the repo.
\item Do a make in the root the repo (make sure you have all u\-Hal, root, boost... libraries on your computer)
\item Launch testpgrm command if you want to test if everything is working good.
\item You can test the data acquisition by lauching datatest2cbc or datatest8cbc.
\item Launch mcp to play with the Test Interface \par
 \par
 {\bfseries What can you do with the software ?}
\end{DoxyEnumerate}

A Glib is created per default (maybe in the future you will be able to play with more than one Glib)

You can then add Modules to the Glib and Cbcs to the Modules you created. When creating a Cbc, you can choose the config file you will load to its register map. If you want to add your personal config file, please make sure to add it in \#define in the H\-W\-Description/\-Description.\-h and then to add an option in the M\-C\-P code.

After this creation round, you can do anything you want \-:
\begin{DoxyItemize}
\item Configure the Glib or the Cbcs
\item Manipulate the registers in the Glib
\item Manipulate the registers in the Cbcs
\item Perform a D\-A\-Q and tracing diagrams counting the number of hits
\item Have a configuration recap of all the objects you created
\end{DoxyItemize}

Concerning the manipulation of the Cbcs, you have the opportunity to modify all the Cbcs of a same Module at once with the Broadcast feature of the Cbc.

When you write a register in the Glib or the Cbc, the writing is updated to the map contained in Glib/\-Cbc objects so that you're always fully aware of what is in the Glib/\-Cbc register.

For writing value in register, we invite you to put in the following format \-: 0x\-\_\-\-\_\-. You must thus type '0x\-F\-F' for exemple and not just 'F\-F' in the command line. That might change in the future..

You can also run D\-A\-Q and recover the Hit Histogram for each Cbc and each Event in the output file.

At the end of the program, the register maps are saved into output\-\_\-\-X\-\_\-\-X.\-txt files you can find in the settings directory. For example, output\-\_\-1\-\_\-2.\-txt contains the register map of the Cbc 2 of the Module 1. You can also find different .pdf files containing the histograms of the D\-A\-Q.

For debugging purpose, you can activate D\-E\-V\-\_\-\-F\-L\-A\-G in files or in Makefile and also activate the u\-Hal log in \hyperlink{_reg_manager_8cc}{Reg\-Manager.\-cc}.

More features will be implemented later, so make sure to have the last release locally to benefit from them.

Warning ! \-: be careful with options choice in the program menus, some mistypes can leed to unexpected hazards \-:-\/(.

\subsection*{On the go... }

~~~~~~{\bfseries Last Updates}


\begin{DoxyItemize}
\item 09/07/14 \-: Added threading for stack writing registers
\item 10/07/14 \-: Read Data from acquisition in a rubbish format
\item $\sim$$\sim$25/07/14 \-: Fully functional for 2\-C\-B\-C (safe from Broadcast obviously), pending for 8\-C\-B\-C$\sim$$\sim$
\item $\sim$$\sim$28/07/14 \-: Found a bug in the reading of C\-B\-C1 of 2\-C\-B\-C, trying to see if coming from soft or hard$\sim$$\sim$
\item 30/07/14 \-: Working 2\-C\-B\-C version, find a 8\-C\-B\-C working version in the 8\-C\-B\-C branch
\item 12/08/14 \-: Working agnostic version of the new structure on Master
\item 15/08/14 \-: System Controller Class working \par
 \par
 {\bfseries Future Improvements}
\item Wrap the project nicely
\end{DoxyItemize}

\subsection*{Support, Suggestions ? }

For any support/suggestions, mail Lorenzo Bidegain, Nicolas Pierre or Georg Auzinger. 